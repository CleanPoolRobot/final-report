\chapter{Introdução}
Neste capítulo, serão descritos o contexto no qual se insere o trabalho e os
objetivos deste trabalho.

\section{Contextualização}
O Brasil é o segundo colocado no ranking mundial de países com maior número de
piscinas, com 1,8 milhão de unidades instaladas (sendo 80\% particulares), atrás
apenas dos Estados Unidos. O país fatura anualmente cerca de R\$ 4,2 bilhões com
a construção e manutenção de piscinas \cite{portalfatorbrasil2013}. Desta forma,
atividades relacionadas ao uso da piscina estão cada vez mais requisitadas, entre
elas a limpeza.

A limpeza do fundo da piscina é um processo que demanda tempo e esforço físico na
sua realização. Uma pesquisa realizada em março de 2014 no Estado de São Paulo
apontou que o valor médio mensal da mão de obra para limpeza de piscinas de até
60 m\textsuperscript{3}, com equipamento exceto produtos para limpeza e manutenção,
foi de R\$ 303,00. Um valor que se torna cada vez mais significativo se imaginarmos
piscinas maiores \cite{datafolha2014}. Com isso, várias empresas vêm investindo
no desenvolvimento de robôs que possam realizar aspiração de piscinas. 

No Brasil existem basicamente 3 tipos de produtos que realizam a tarefa de limpeza:
aspiradores manuais que são os mais comuns no mercado, aspiradores robóticos,
totalmente automatizados porém com alto custo, e aspiradores autônomos porém pouco
eficazes \cite{miura2006}.

Considerando o alto custo de aquisição de equipamentos modernos, internacionais
e a grande demanda por serviços de higienização de piscinas, e o fato de que
no Brasil não há empresas que desenvolvam este produto (há apenas revendedoras),
o desenvolvimento do \textit{Clean Pool Robot} (\textsf{CPR}) permitirá a
aplicação do conceito de automação a ambientes residenciais e a redução dos
gastos com a contratação de um serviço terceirizado. As principais vantagens
na utilização do \textit{Clean Pool Robot} são:

\begin{itemize}
\item Retirar detritos do fundo da piscina que não foram alcançados por outros
meios;
\item Mover a água ao passo que limpa as superfícies, melhorando a sua circulação;
\item Produzir um produto nacional que possui valor de mercado mais acessível.
\end{itemize}

\section{Objetivos}
Esse trabalho visou alcançar os objetivos, Geral e Específicos, apresentados a seguir.

\subsection{Objetivo Geral}
Construir um robô capaz de realizar a remoção de sujeiras depositadas no fundo
de piscinas por meio da aspiração e filtragem. Entende-se como sujeira a ser
removida folhas, algas, grãos de terra e gravetos.

\subsection{Objetivos Específicos}
\begin{itemize}
\item Realizar aspiração automática dos resíduos decantados por meio da sucçãoe filtragem;
\item Submergir de forma independente;
\item Movimentar-se ao longo do fundo da piscina.
\end{itemize}